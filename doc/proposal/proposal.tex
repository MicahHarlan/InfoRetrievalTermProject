\documentclass[journal]{IEEEtran}

\usepackage{cite}
\usepackage{authblk}
\usepackage[pdftex]{graphicx}
\usepackage{amsmath}
\usepackage{algorithmic}
\usepackage{array}
\usepackage{url}

% correct bad hyphenation here
\hyphenation{op-tical net-works semi-conduc-tor}

\begin{document}

% paper title
\title{Badass Title}

% author names and affiliations
\author[1]{Micah Keith Harlan}
\author[1]{Jianger Yu}
\author[1]{Baiyi Zhang}

% For affiliations, directly adjust the font size using \small, \footnotesize, etc.
\affil[1]{{\small \{mharlan25, jianger, baiyi\}@vt.edu}}
\affil[1]{{\small Department of Computer Science, Virginia Polytechnic Institute and State University, Falls Church, VA, USA}}

% make the title area
\maketitle

% abstract
\begin{abstract}
The rise of streaming services has revolutionized the way we consume media, offering an unprecedented volume of content at our fingertips. However, this abundance has led to a paradox of choice, where users often find themselves overwhelmed, scrolling through streaming platforms for extended periods without being able to decide on a movie to watch. Traditional search mechanisms in these platforms typically hinge on metadata such as movie titles, genres, and actor names, which, despite their utility, often need to catch up in catering to the nuanced preferences of users. By incorporating Large Language Models (LLMs), it can offer a more nuanced search capability, allowing users to find movies based on detailed aspects of the content, such as thematic elements, narrative style, or emotional tone, far beyond what conventional metadata can provide. This paper aims to provide an innovative way in regards to query and recommend movies to users, by providing an interactive application, where a user can query for, and get recommended movies to reduce user browsing time.
\end{abstract}

% Note that keywords are not normally used for peerreview papers.
\begin{IEEEkeywords}
IEEE, journal, \LaTeX, paper, template.
\end{IEEEkeywords}

\section{Introduction}
% The very first letter is a 2 line initial drop letter followed by the rest of the first word in caps.
\IEEEPARstart{T}{his} is a test\cite{manning2008introduction}.

\section{Conclusion}
The conclusion goes here.

% references section
\bibliographystyle{IEEEtran}
\bibliography{IEEEabrv, proposal-ref}

\end{document}